%!TEX root = ../report.tex
\documentclass[../report.tex]{subfiles}

\begin{document}
    \begin{abstract}
        Forests present a highly heterogeneous environment composed of terrain, vegetation, tree trunks, tree stumps, and other natural elements. The irregularity of Earth’s topography, combined with dense vegetation and occlusions, poses a significant challenge for distinguishing terrain from other environmental components—particularly when using traditional machine learning approaches. A major bottleneck in this task lies in the feature extraction process from noisy and partially labeled 3D point cloud datasets, which often lack semantic consistency and spatial clarity. This report proposes a robust and adaptable pipeline for semantic segmentation of 3D point clouds, using 2D annotations to guide learning. Our method integrates state-of-the-art feature extraction techniques and clustering algorithms to isolate tree stumps and terrain as independent semantic clusters. The proposed approach achieves a segmentation accuracy of 91.3\% for terrain classification and 88.7\% for tree stump detection across a benchmark forest UAV dataset containing over 2.5 million points. Furthermore, it demonstrates resilience to noise and generalizes well across different terrains. The outcomes of this work are highly applicable to UAV-based forest monitoring, with potential extensions to vegetation indexing, biomass estimation, and land-use analysis in both forested and urban settings. The pipeline not only addresses the limitations of existing methods but also sets a strong foundation for scalable, real-time environmental perception in ecological research and smart forestry initiatives.
        
        \textbf{Keywords:}Forest segmentation, Point cloud processing, Tree stump detection, Terrain classification, UAV-based monitoring, Machine vision, Terrain Cluster, Object segmentation, 2D labeled point clouds, Machine learning in forestry, PointNet, PointNet2.
    \end{abstract}
\end{document}
